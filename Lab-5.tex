% Options for packages loaded elsewhere
\PassOptionsToPackage{unicode}{hyperref}
\PassOptionsToPackage{hyphens}{url}
%
\documentclass[
]{article}
\usepackage{lmodern}
\usepackage{amssymb,amsmath}
\usepackage{ifxetex,ifluatex}
\ifnum 0\ifxetex 1\fi\ifluatex 1\fi=0 % if pdftex
  \usepackage[T1]{fontenc}
  \usepackage[utf8]{inputenc}
  \usepackage{textcomp} % provide euro and other symbols
\else % if luatex or xetex
  \usepackage{unicode-math}
  \defaultfontfeatures{Scale=MatchLowercase}
  \defaultfontfeatures[\rmfamily]{Ligatures=TeX,Scale=1}
  \setmonofont[]{Times New Roman}
\fi
% Use upquote if available, for straight quotes in verbatim environments
\IfFileExists{upquote.sty}{\usepackage{upquote}}{}
\IfFileExists{microtype.sty}{% use microtype if available
  \usepackage[]{microtype}
  \UseMicrotypeSet[protrusion]{basicmath} % disable protrusion for tt fonts
}{}
\makeatletter
\@ifundefined{KOMAClassName}{% if non-KOMA class
  \IfFileExists{parskip.sty}{%
    \usepackage{parskip}
  }{% else
    \setlength{\parindent}{0pt}
    \setlength{\parskip}{6pt plus 2pt minus 1pt}}
}{% if KOMA class
  \KOMAoptions{parskip=half}}
\makeatother
\usepackage{xcolor}
\IfFileExists{xurl.sty}{\usepackage{xurl}}{} % add URL line breaks if available
\IfFileExists{bookmark.sty}{\usepackage{bookmark}}{\usepackage{hyperref}}
\hypersetup{
  pdftitle={Lab 5},
  pdfauthor={Kristina Arevalo},
  hidelinks,
  pdfcreator={LaTeX via pandoc}}
\urlstyle{same} % disable monospaced font for URLs
\usepackage[margin=1in]{geometry}
\usepackage{color}
\usepackage{fancyvrb}
\newcommand{\VerbBar}{|}
\newcommand{\VERB}{\Verb[commandchars=\\\{\}]}
\DefineVerbatimEnvironment{Highlighting}{Verbatim}{commandchars=\\\{\}}
% Add ',fontsize=\small' for more characters per line
\usepackage{framed}
\definecolor{shadecolor}{RGB}{248,248,248}
\newenvironment{Shaded}{\begin{snugshade}}{\end{snugshade}}
\newcommand{\AlertTok}[1]{\textcolor[rgb]{0.94,0.16,0.16}{#1}}
\newcommand{\AnnotationTok}[1]{\textcolor[rgb]{0.56,0.35,0.01}{\textbf{\textit{#1}}}}
\newcommand{\AttributeTok}[1]{\textcolor[rgb]{0.77,0.63,0.00}{#1}}
\newcommand{\BaseNTok}[1]{\textcolor[rgb]{0.00,0.00,0.81}{#1}}
\newcommand{\BuiltInTok}[1]{#1}
\newcommand{\CharTok}[1]{\textcolor[rgb]{0.31,0.60,0.02}{#1}}
\newcommand{\CommentTok}[1]{\textcolor[rgb]{0.56,0.35,0.01}{\textit{#1}}}
\newcommand{\CommentVarTok}[1]{\textcolor[rgb]{0.56,0.35,0.01}{\textbf{\textit{#1}}}}
\newcommand{\ConstantTok}[1]{\textcolor[rgb]{0.00,0.00,0.00}{#1}}
\newcommand{\ControlFlowTok}[1]{\textcolor[rgb]{0.13,0.29,0.53}{\textbf{#1}}}
\newcommand{\DataTypeTok}[1]{\textcolor[rgb]{0.13,0.29,0.53}{#1}}
\newcommand{\DecValTok}[1]{\textcolor[rgb]{0.00,0.00,0.81}{#1}}
\newcommand{\DocumentationTok}[1]{\textcolor[rgb]{0.56,0.35,0.01}{\textbf{\textit{#1}}}}
\newcommand{\ErrorTok}[1]{\textcolor[rgb]{0.64,0.00,0.00}{\textbf{#1}}}
\newcommand{\ExtensionTok}[1]{#1}
\newcommand{\FloatTok}[1]{\textcolor[rgb]{0.00,0.00,0.81}{#1}}
\newcommand{\FunctionTok}[1]{\textcolor[rgb]{0.00,0.00,0.00}{#1}}
\newcommand{\ImportTok}[1]{#1}
\newcommand{\InformationTok}[1]{\textcolor[rgb]{0.56,0.35,0.01}{\textbf{\textit{#1}}}}
\newcommand{\KeywordTok}[1]{\textcolor[rgb]{0.13,0.29,0.53}{\textbf{#1}}}
\newcommand{\NormalTok}[1]{#1}
\newcommand{\OperatorTok}[1]{\textcolor[rgb]{0.81,0.36,0.00}{\textbf{#1}}}
\newcommand{\OtherTok}[1]{\textcolor[rgb]{0.56,0.35,0.01}{#1}}
\newcommand{\PreprocessorTok}[1]{\textcolor[rgb]{0.56,0.35,0.01}{\textit{#1}}}
\newcommand{\RegionMarkerTok}[1]{#1}
\newcommand{\SpecialCharTok}[1]{\textcolor[rgb]{0.00,0.00,0.00}{#1}}
\newcommand{\SpecialStringTok}[1]{\textcolor[rgb]{0.31,0.60,0.02}{#1}}
\newcommand{\StringTok}[1]{\textcolor[rgb]{0.31,0.60,0.02}{#1}}
\newcommand{\VariableTok}[1]{\textcolor[rgb]{0.00,0.00,0.00}{#1}}
\newcommand{\VerbatimStringTok}[1]{\textcolor[rgb]{0.31,0.60,0.02}{#1}}
\newcommand{\WarningTok}[1]{\textcolor[rgb]{0.56,0.35,0.01}{\textbf{\textit{#1}}}}
\usepackage{graphicx,grffile}
\makeatletter
\def\maxwidth{\ifdim\Gin@nat@width>\linewidth\linewidth\else\Gin@nat@width\fi}
\def\maxheight{\ifdim\Gin@nat@height>\textheight\textheight\else\Gin@nat@height\fi}
\makeatother
% Scale images if necessary, so that they will not overflow the page
% margins by default, and it is still possible to overwrite the defaults
% using explicit options in \includegraphics[width, height, ...]{}
\setkeys{Gin}{width=\maxwidth,height=\maxheight,keepaspectratio}
% Set default figure placement to htbp
\makeatletter
\def\fps@figure{htbp}
\makeatother
\setlength{\emergencystretch}{3em} % prevent overfull lines
\providecommand{\tightlist}{%
  \setlength{\itemsep}{0pt}\setlength{\parskip}{0pt}}
\setcounter{secnumdepth}{-\maxdimen} % remove section numbering

\title{Lab 5}
\author{Kristina Arevalo}
\date{Fri Mar 19 2021}

\begin{document}
\maketitle

{
\setcounter{tocdepth}{2}
\tableofcontents
}
\newpage

\hypertarget{problem-1}{%
\section{Problem 1}\label{problem-1}}

In the context of making decisions about whether or not observed data is
consistent or inconsistent with a Null hypothesis, it is possible to
make errors. The questions below ask you to create simulated data
containing patterns that could lead to correct and incorrect decisions
about the role of the null hypothesis.

Consider a design with 3 groups, and 10 people per group. Assume that
the dependent variable is assumed to be normally distributed, and use
unit normal distributions with mean = 0, and sd = 1 in your simulations.

Create simulated data for the above design that could be produced by the
null hypothesis, and that results in a 𝐹 value that is smaller than the
critical value for 𝐹 in this design (assume alpha = .05). Report the
ANOVA, and show a ggplot of the means in the simulated data.
Furthermore, display the individual data points on top of the means.
Would you reject the null hypothesis in this situation, and would you be
incorrect or correct in rejecting the null? (3 points)

\hypertarget{simulated-data}{%
\subsection{simulated data}\label{simulated-data}}

\begin{Shaded}
\begin{Highlighting}[]
\NormalTok{levels <-}\StringTok{ }\DecValTok{3}
\NormalTok{n_per_level <-}\StringTok{ }\DecValTok{10}
\NormalTok{crit_F <-}\StringTok{ }\KeywordTok{qf}\NormalTok{(.}\DecValTok{95}\NormalTok{,}\DecValTok{2}\NormalTok{,}\DecValTok{27}\NormalTok{)}

\ControlFlowTok{for}\NormalTok{(i }\ControlFlowTok{in} \DecValTok{1}\OperatorTok{:}\DecValTok{1000}\NormalTok{)\{}
\NormalTok{  sim_data <-}\StringTok{ }\KeywordTok{tibble}\NormalTok{(}\DataTypeTok{participants =} \DecValTok{1}\OperatorTok{:}\NormalTok{(levels}\OperatorTok{*}\NormalTok{n_per_level),}
                     \DataTypeTok{IV =} \KeywordTok{as.factor}\NormalTok{(}\KeywordTok{rep}\NormalTok{(}\DecValTok{1}\OperatorTok{:}\NormalTok{levels, }\DataTypeTok{each =}\NormalTok{ n_per_level)),}
                     \DataTypeTok{DV =} \KeywordTok{rnorm}\NormalTok{(levels}\OperatorTok{*}\NormalTok{n_per_level,}\DecValTok{0}\NormalTok{, }\DecValTok{1}\NormalTok{))}
  
\NormalTok{  my.aov <-}\StringTok{ }\KeywordTok{aov}\NormalTok{(DV}\OperatorTok{~}\NormalTok{IV, }\DataTypeTok{data =}\NormalTok{ sim_data) }\OperatorTok\StringTok{ }\KeywordTok{summary}\NormalTok{()}
\NormalTok{  simulated_F <-}\StringTok{ }\NormalTok{my.aov[[}\DecValTok{1}\NormalTok{]]}\OperatorTok{$}\StringTok{`}\DataTypeTok{F value}\StringTok{`}\NormalTok{[}\DecValTok{1}\NormalTok{]}
  
  
  \ControlFlowTok{if}\NormalTok{(simulated_F }\OperatorTok{<}\StringTok{ }\NormalTok{crit_F) }\ControlFlowTok{break} 
\NormalTok{\}}
\end{Highlighting}
\end{Shaded}

\#anova

\begin{Shaded}
\begin{Highlighting}[]
\NormalTok{my.aov}
\end{Highlighting}
\end{Shaded}

\begin{verbatim}
##             Df Sum Sq Mean Sq F value Pr(>F)
## IV           2  0.117  0.0586   0.057  0.945
## Residuals   27 27.673  1.0249
\end{verbatim}

\hypertarget{ggplot}{%
\subsection{ggplot}\label{ggplot}}

\begin{Shaded}
\begin{Highlighting}[]
\KeywordTok{ggplot}\NormalTok{(sim_data, }\KeywordTok{aes}\NormalTok{(}\DataTypeTok{x =}\NormalTok{ IV, }\DataTypeTok{y=}\NormalTok{ DV))}\OperatorTok{+}
\StringTok{  }\KeywordTok{geom_bar}\NormalTok{(}\DataTypeTok{stat=} \StringTok{"summary"}\NormalTok{, }\DataTypeTok{fun =} \StringTok{"mean"}\NormalTok{)}\OperatorTok{+}
\StringTok{  }\KeywordTok{geom_point}\NormalTok{()}
\end{Highlighting}
\end{Shaded}

\includegraphics{Lab-5_files/figure-latex/unnamed-chunk-4-1.pdf}

I would fail to reject the null and I believe I would be correct in
doing so because the p-value of obtaining this F-value (or smaller) is
not less than .05 and we assumed an alpha of .05. These results are not
significant.

\hypertarget{problem-2}{%
\section{Problem 2}\label{problem-2}}

Create simulated data for the above design that could be produced by the
null hypothesis, and that results in a 𝐹 value that is larger than the
critical value for 𝐹 in this design (assume alpha = .05). Report the
ANOVA, and show a ggplot of the means in the simulated data.
Furthermore, display the individual data points on top of the means.
Would you reject the null hypothesis in this situation, and would you be
incorrect or correct in rejecting the null? (3 points)

\hypertarget{simulated-data-1}{%
\subsection{simulated data}\label{simulated-data-1}}

\begin{Shaded}
\begin{Highlighting}[]
\NormalTok{levels <-}\StringTok{ }\DecValTok{3}
\NormalTok{n_per_level <-}\StringTok{ }\DecValTok{10}
\NormalTok{crit_F <-}\StringTok{ }\KeywordTok{qf}\NormalTok{(.}\DecValTok{95}\NormalTok{,}\DecValTok{2}\NormalTok{,}\DecValTok{27}\NormalTok{)}

\ControlFlowTok{for}\NormalTok{(i }\ControlFlowTok{in} \DecValTok{1}\OperatorTok{:}\DecValTok{1000}\NormalTok{)\{}
\NormalTok{  sim_data <-}\StringTok{ }\KeywordTok{tibble}\NormalTok{(}\DataTypeTok{participants =} \DecValTok{1}\OperatorTok{:}\NormalTok{(levels}\OperatorTok{*}\NormalTok{n_per_level),}
                     \DataTypeTok{IV =} \KeywordTok{as.factor}\NormalTok{(}\KeywordTok{rep}\NormalTok{(}\DecValTok{1}\OperatorTok{:}\NormalTok{levels, }\DataTypeTok{each =}\NormalTok{ n_per_level)),}
                     \DataTypeTok{DV =} \KeywordTok{rnorm}\NormalTok{(levels}\OperatorTok{*}\NormalTok{n_per_level,}\DecValTok{0}\NormalTok{, }\DecValTok{1}\NormalTok{))}
  
\NormalTok{  my.aov <-}\StringTok{ }\KeywordTok{aov}\NormalTok{(DV}\OperatorTok{~}\NormalTok{IV, }\DataTypeTok{data =}\NormalTok{ sim_data) }\OperatorTok\StringTok{ }\KeywordTok{summary}\NormalTok{()}
\NormalTok{  simulated_F <-}\StringTok{ }\NormalTok{my.aov[[}\DecValTok{1}\NormalTok{]]}\OperatorTok{$}\StringTok{`}\DataTypeTok{F value}\StringTok{`}\NormalTok{[}\DecValTok{1}\NormalTok{]}
  
  
  \ControlFlowTok{if}\NormalTok{(simulated_F }\OperatorTok{>}\StringTok{ }\NormalTok{crit_F) }\ControlFlowTok{break} 
\NormalTok{\}}
\end{Highlighting}
\end{Shaded}

\#anova

\begin{Shaded}
\begin{Highlighting}[]
\NormalTok{my.aov}
\end{Highlighting}
\end{Shaded}

\begin{verbatim}
##             Df Sum Sq Mean Sq F value Pr(>F)  
## IV           2  7.645   3.823   4.518 0.0203 *
## Residuals   27 22.842   0.846                 
## ---
## Signif. codes:  0 '***' 0.001 '**' 0.01 '*' 0.05 '.' 0.1 ' ' 1
\end{verbatim}

\hypertarget{ggplot-1}{%
\subsection{ggplot}\label{ggplot-1}}

\begin{Shaded}
\begin{Highlighting}[]
\KeywordTok{ggplot}\NormalTok{(sim_data, }\KeywordTok{aes}\NormalTok{(}\DataTypeTok{x =}\NormalTok{ IV, }\DataTypeTok{y=}\NormalTok{ DV))}\OperatorTok{+}
\StringTok{  }\KeywordTok{geom_bar}\NormalTok{(}\DataTypeTok{stat=} \StringTok{"summary"}\NormalTok{, }\DataTypeTok{fun =} \StringTok{"mean"}\NormalTok{)}\OperatorTok{+}
\StringTok{  }\KeywordTok{geom_point}\NormalTok{()}
\end{Highlighting}
\end{Shaded}

\includegraphics{Lab-5_files/figure-latex/unnamed-chunk-7-1.pdf} I would
reject the null because my p-value is less than .05. I would be correct
in doing so because my p-value is less than my alpha level. These
results would be considered significant.

\hypertarget{bonus-question}{%
\subsection{Bonus Question}\label{bonus-question}}

In the lab we saw that F-distribution is robust to violations of the
assumptions of ANOVA. For example, the simulation of the null based on a
bi-modal distribution was very similar to the true F distribution. For
this bonus question, show that you can ``break'' the F-distribution.
Specifically, can you run a simulation that samples numbers from a
non-normal distribution that does produce a very different looking
F-distribution? (3 points)

\begin{Shaded}
\begin{Highlighting}[]
\NormalTok{levels <-}\StringTok{ }\DecValTok{3}
\NormalTok{n_per_level <-}\StringTok{ }\DecValTok{10}

\NormalTok{save_F_values <-}\StringTok{ }\KeywordTok{length}\NormalTok{(}\DecValTok{1000}\NormalTok{)}
\ControlFlowTok{for}\NormalTok{(i }\ControlFlowTok{in} \DecValTok{1}\OperatorTok{:}\DecValTok{1000}\NormalTok{)\{}
\NormalTok{random_data <-}\StringTok{ }\KeywordTok{tibble}\NormalTok{(}\DataTypeTok{subjects =} \DecValTok{1}\OperatorTok{:}\NormalTok{(levels}\OperatorTok{*}\NormalTok{n_per_level),}
                      \DataTypeTok{IV =} \KeywordTok{as.factor}\NormalTok{(}\KeywordTok{rep}\NormalTok{(}\DecValTok{1}\OperatorTok{:}\NormalTok{levels, }\DataTypeTok{each =}\NormalTok{ n_per_level)),}
                      \DataTypeTok{DV =} \KeywordTok{rnorm}\NormalTok{(levels}\OperatorTok{*}\NormalTok{n_per_level, }\DecValTok{0}\NormalTok{, }\DecValTok{1}\NormalTok{)}
\NormalTok{                      )}
\NormalTok{aov.out <-}\StringTok{ }\KeywordTok{aov}\NormalTok{(DV }\OperatorTok{~}\StringTok{ }\NormalTok{IV, }\DataTypeTok{data =}\NormalTok{ random_data)}
\NormalTok{simulated_F <-}\StringTok{ }\KeywordTok{summary}\NormalTok{(aov.out)[[}\DecValTok{1}\NormalTok{]]}\OperatorTok{$}\StringTok{`}\DataTypeTok{F value}\StringTok{`}\NormalTok{[}\DecValTok{1}\NormalTok{]}
\NormalTok{save_F_values[i] <-}\StringTok{ }\NormalTok{simulated_F}
\NormalTok{\}}

\NormalTok{F_comparison <-}\StringTok{ }\KeywordTok{tibble}\NormalTok{(}\DataTypeTok{type =} \KeywordTok{rep}\NormalTok{(}\KeywordTok{c}\NormalTok{(}\StringTok{"analytic"}\NormalTok{,}\StringTok{"simulated"}\NormalTok{), }\DataTypeTok{each =} \DecValTok{1000}\NormalTok{),}
                        \DataTypeTok{F_value =} \KeywordTok{c}\NormalTok{(}\KeywordTok{rf}\NormalTok{(}\DecValTok{1000}\NormalTok{,levels}\DecValTok{-1}\NormalTok{,levels}\OperatorTok{*}\NormalTok{n_per_level),save_F_values))}

\KeywordTok{ggplot}\NormalTok{(F_comparison, }\KeywordTok{aes}\NormalTok{(}\DataTypeTok{x=}\NormalTok{F_value, }\DataTypeTok{color =}\NormalTok{ type))}\OperatorTok{+}
\StringTok{  }\KeywordTok{geom_freqpoly}\NormalTok{(}\DataTypeTok{bins =} \DecValTok{50}\NormalTok{)}
\end{Highlighting}
\end{Shaded}

\includegraphics{Lab-5_files/figure-latex/unnamed-chunk-8-1.pdf}

\end{document}
